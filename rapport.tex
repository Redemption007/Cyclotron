\documentclass[a4paper,11pt]{article}			% Définit le type de document

\usepackage{lmodern}							% Police vectorielle

\usepackage[frenchb]{babel}						% Définit la langue du document
\usepackage[utf8]{inputenc}						% Encodage d'entrée : (é, û, ö, ...)
\usepackage[T1]{fontenc}						% Encodage de sortie (affichage)

\usepackage{amsmath,amsfonts,amssymb}			% Utilisation de commandes, polices et symboles de maths
\usepackage{graphicx}							% Importation d'images

\usepackage[left=2.5cm,right=2.5cm,top=2.5cm,bottom=2.5cm]{geometry}	% Modification des marges

\usepackage{bm}									% commande \bm (met le texte en gras dans les equations)





\begin{document}

\vspace*{-2cm}

\centerline{\LARGE Faculté des Sciences et des Techniques}
  \vspace*{0.5cm}  
\centerline{\LARGE X22P050 : Modélisation pour Physique  2 }  
\vspace*{0.5cm}  

\centerline{\bf\Large Titre projet :}
\vspace*{2cm}
\noindent
\noindent{\bf Nom 1 et prénom 1 :   }\\   \\
\vspace*{2cm}
\noindent{\bf Nom 2 et prénom 2 :   }\\   \\


\section*{Résumé :}


\end{document}
